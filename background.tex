\mychapter{1}{Background}
\section{The necessity of Carbon Dioxide Removal}
\textcite{Bergman2021TheJustice} estimate that human activities have caused 2500 Gt\footnote{ 1 Gt = 1 billion (metric) tons} of carbon emissions into the atmosphere since 1850. This has caused the atmospheric concentration of CO\textsubscript{2} to rise from 278 ppm before the beginning of the industrial era to over 410 ppm in 2021 \parencite[5]{Friedlingstein2022Global2022}. Anthropogenic global warming already reached 1 degree Celsius in 2017 \parencite[6]{IPCC2018Global1.5C}, and the \textcite[30]{UNEP2022Emissions2022} estimates that even in the most optimistic scenarios, peak warming will likely exceed the 1.5-degree goals.\footnote{The Mercator Institute estimates that as of May 8, 2023, the remaining carbon budget to stay below the 1.5-degree goal is roughly 262Gt CO\textsubscript{2}, which at the current pace will be exceeded within the next seven yerars. The current numbers can be found at: https://www.mcc-berlin.net/en/research/co2-budget.html} Similarly, a report by the \textcite[9]{NationalAcademiesofSciences2018NegativeAgenda} concludes that to meet the goals of the Paris Agreement, negative emissions of 10 Gt/y CO\textsubscript{2} by 2050 and 20 Gt/y CO\textsubscript{2} by 2100 are required.\\
Additionally, while for most sectors and industries the cost of CO\textsubscript{2} abatement through either emission avoidance or capture is significantly lower than the projected cost of most CDR strategies, there are some hard-to-avoid emissions, e.g. from agriculture and air transportation. This means that even after an atmospheric concentration of CO\textsubscript{2} in accordance with the goals of the Paris Agreement is reached, some carbon removal will be required to offset the remaining emissions and ensure the concentration of CO\textsubscript{2} remains stable \parencite{NationalResearchCouncil2015ClimateSequestration}.\footnote{The amount of CDR required depends on the definition of \textit{hard-to-avoid emissions}. While many definitions include any emissions for which their high financial cost makes CO\textsubscript{2} abatement non-viable, Bergman and Rinberg only include emissions that cannot be avoided for physical reasons or whose avoidance would cause unacceptable social injustice, and warn against using proposed future CDR strategies as an argument to avoid emission reduction in the present.}
\section{Definition of Carbon Dioxide Removal}
There is a number of different definitions of Carbon Dioxide Removal (CDR), all of which are broadly equal, but slightly differ in the scope of strategies and technologies they include. For example, the \textcite[544]{IPCC2018Global1.5C} defines CDR as "anthropogenic activities removing CO\textsubscript{2} from the atmosphere and durably storing it in geological, terrestrial or ocean reservoirs, or in products", which suggests that any kind of carbon capture or storage constitutes CDR, given that the carbon was taken from the atmosphere and not a single point source. The \textcite[1]{NationalAcademiesofSciences2018NegativeAgenda} use the term Negative Emissions Technologies (NETs) instead, which they define as strategies "which remove carbon from the atmosphere and sequester it." More recently, \textcite[11]{Smith2023TheEdition} argued that CDR must follow the three principles of (1) removing carbon from the atmosphere, (2) using durable\footnote{ Smith et. al define durable storage as storage that "has a characteristic timescale on the order of decades or more".} storage, and (3) being the result of active human intervention. All of these definitions imply that CDR can be divided into approaches that capture CO\textsubscript{2} from the atmosphere, such as Direct Air Capture, approaches that store CO\textsubscript{2}, such as geological storage or in-situ mineralization, and approaches that capture and store CO\textsubscript{2}, such as land management or afforestation and reforestation. This thesis uses the more narrow definition, focusing only on strategies and technologies that either capture or capture and store carbon, excluding storage-only strategies.
\section{Natural and technological CDR strategies}
CDR strategies can be grouped into biological capture, geochemical capture, both of which mostly rely on using or enhancing natural (i.e. biological or chemical) processes, and chemical approaches, which uses technological processes involving chemical solvents \parencite{Smith2023TheEdition}.\\
Biological capture processes include land-management strategies that aim to increase the amount of CO\textsubscript{2} stored in e.g. agriculturally-used soil, afforestation and reforestation, which are based on the natural ability of trees to capture and sequester CO\textsubscript{2} during their growth phase and beyond, as well as sea-based approaches that rely on fertilizing oceans to increase the growth of phytoplankton or seaweed.\\
Geochemical capture processes are mostly based on enhancing the natural weathering and mineralization process, which binds carbon in solid rock material, as well as increasing the alkalinity of oceans, which increases their ability to absorb CO\textsubscript{2}.\\
Chemical processes use novel, typically two-stage technological processes that involve solid or liquid solvents, which initially absorb CO\textsubscript{2} from the atmosphere and later release it in concentrated form for storage or sequestration.\\
The various approaches differ greatly in their CO\textsubscript{2} removal potential, their feasibility on a larger scale, their energy requirements, the durability of the storage they use, and their cost, measured in dollars per ton of CO\textsubscript{2} removed \parencite[Chapter 8]{NationalAcademiesofSciences2018NegativeAgenda}.\\The approaches discussed in this thesis are largely those that were considered by the National Academies in their 2018 research agenda.
\section{Current State of CDR research and development}
In the past years, the investment into CDR research and development of CDR projects has increased substantially \parencite[11]{Smith2023TheEdition}. However, research and development of the various CDR strategies are currently at very different stages. Biological processes have a long history of literature and are more well-understood, and they are also cheaper to deploy at the present time, but less durable and face greater limitations in their total carbon removal potential due to natural constraints \parencite{Fuss2018NegativeEffects}. In comparison, chemical processes are still in their infancy and limited to few, small-scale experiments \parencite[24]{Smith2023TheEdition}. For example, while direct air capture is believed to have a carbon removal potential in the Gt/y range, it presently faces issues of high costs and lack of experience apart from relatively small-scale experiments. The largest currently operational direct air capture plant captures 4 kt of CO\textsubscript{2} annually, and even the largest project currently under development will only capture 1 Mt of CO\textsubscript{2} per year \parencite{Pernot2022DirectNeeded}. \\
Uncertainty also exists about the contribution CDR can make to climate change mitigation efforts in general. While some reports (e.g. \textcite{NationalAcademiesofSciences2018NegativeAgenda}) include the usage of large amounts of CDR in the future as a necessity to reach the goals of the Paris Agreement, \textcite{Fuss2018NegativeEffects} argue that CDR is no alternative to rapid and sustainable reductions in carbon emissions and warn against over-reliance on CDR, citing the uncertainty and risks involved in its large-scale deployment.\\
In summary, while the efforts to develop and deploy strategies for anthropogenic carbon removal are increasing and many different approaches are being trialed, there is still a lot of research to be done before CDR can be used on a scale large enough to make a significant impact on climate change. Until then, efforts to reduce carbon emissions and reach a net zero should not only be continued but increased.
\section{Economic viability of CDR}
Since carbon removal is generally (with some exceptions) costly, and comes with no immediate benefit to its producer, the economic viability of CDR depends on a carbon price instituted by systems such as the European Emissions Trading System (EU-ETS), which requires companies in some industries to buy carbon certificates and allows trading between companies, or a government-enforced carbon tax. Currently (May 2023), the price for one ton of carbon dioxide produced in the EU-ETS stands at 88 € (97 US\$), after exceeding 100 € for the first time in February of this year. Other definitions of carbon costs include the total cost of the social damage caused by carbon emissions, with different models proposing prices in a wide range between 50 US\$ and 5000 US\$ \parencite[1,7]{Kikstra2021TheVariability}. This thesis uses the definition instituted by \textcite[3]{Griscom2017NaturalSolutions}, who define a price of 100 US\$/t CO\textsubscript{2} as cost-effective based on the costs to society avoided if global warming is kept to a level below two degrees Celsius.
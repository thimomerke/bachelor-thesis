\mychapter{4}{4. Chemical Strategies}
\section{Direct Air Capture}
In contrast to biological and geochemical approaches, which are based on enhancing natural processes, Direct Air Capture (DAC) is a novel approach to carbon dioxide removal that utilizes engineered chemical reactions to remove CO\textsubscript{2} directly from the ambient air. DAC uses a filtration system to capture CO\textsubscript{2} molecules and generate a concentrated stream of CO\textsubscript{2} for sequestration or sale as a commodity. This process can use liquid solvents, such as calcium hydroxide (LDAC) or solid sorbents, such as amines (SDAC). The solvent or sorbent is then regenerated using heat to release captured CO\textsubscript{2} in concentrated form for further processing or storage \parencite{Gorman2021CarbonHerzog, Mulligan2020CarbonShot:States}.
\subsection*{Potential}
The main advantage of DAC over the approaches discussed above is that it does not require large amounts of land or continuous supplies of reactive material \parencite{Mulligan2020CarbonShot:States, IEA2022DirectZero}.
Generally speaking, LDAC is more suitable for continuous operations in large-scale industrial settings, and the proposed facilities are comparable in size to power plants. SDAC is more suitable for smaller-scale batch applications due to its modular nature and the need to remove sorbents for regeneration \parencite[22-23]{IEA2022DirectZero}. The efficiency and amount of carbon removed through DAC is easy to measure and quantify, as it produces a highly concentrated stream of CO\textsubscript{2} \parencite{Lawler2022WhatTechnology}.
The global potential of DAC as a CDR technology is significant, as it is limited only by the available funding to construct and maintain the facilities. The \textcite{IEA2022DirectZero} expects an initially slow scale-up, with only 85Mt/y in CO\textsubscript{2} removal capacity deployed by 2030, but a significant increase in pace throughout the next decades. Most studies believe that by 2050, 1 to 2 Gt/y could be in operation, and by 2100, DAC could account for 10 to 15 Gt/y \parencite{Mulligan2020CarbonShot:States, IEA2022DirectZero}, providing a substantial portion of the total required annual negative CO\textsubscript{2} emissions. The highest hypothetical calculations even assume that up to 40 Gt/y are possible \parencite{Fuss2018NegativeEffects}.
Due to its independence from any space or material requirement, DAC is inherently flexible and can be placed anywhere, making it suitable for various industrial or geographical contexts \parencite[189]{NAS2018NegativeAgenda}. For example, DAC plants could be placed close to the cheapest available energy sources, leading to reduced energy costs and increased economic viability. Other proposals include the conversion of old industrial plants, such as refineries, into DAC facilities to use the existing energy and pipeline infrastructure. In cities, skyscrapers could be fitted with DAC technology due to their large surface area and the potential to realize co-benefits by, for example, using filters to also remove air pollution or capture water from the air \parencite{Lawler2022WhatTechnology}. Some studies also argue that CO\textsubscript{2} captured by DAC can be provided at any desired concentration and could be used in various commercial applications, such as carbonation of beverages and enhanced oil recovery. However, the commercial potential is limited, and the use of CO\textsubscript{2} removed from the atmosphere to extract more oil would ultimately be counterproductive in terms of mitigating climate change \parencite{Erans2022DirectChallenges}.

\subsection*{Limitations}
DAC is currently deployed only on a very small scale. In total, 18 facilities of varying sizes and capacities from 1 t/y to 4 kt/y are in operation globally, with a total capacity of 8 kt/yr \parencite[18]{IEA2022DirectZero}. The largest single plant is the "Orca" facility run by the Swiss company Climeworks in Iceland, which removes 4 kt/y of CO\textsubscript{2} from the air using a combination of renewable energy and solid sorbent technology. This means that the feasibility and potential problems of large-scale deployment are not yet fully understood.
Despite the promising potential of DAC as a CDR approach, there are several limitations to its implementation. The root cause of these limitations lies in the low concentration of CO\textsubscript{2} in the ambient air, compared to point sources such as fossil fuel power plants, making the capture of CO\textsubscript{2} more difficult and requires highly binding solvents or sorbents. In turn, regenerating liquid solvents requires extremely high temperatures of around 900 °C \parencite[23]{IEA2022DirectZero}. Solid sorbents require less heat or can even be regenerated by vacuum or pressure swing adsorption, but often have lower capture efficiencies than liquid solvents \parencite[192]{NAS2018NegativeAgenda}.
LDAC requires approximately as much energy to remove 1 t CO\textsubscript{2} as was made usable when 1 t CO\textsubscript{2} was released during the burning of fossil fuels. SDAC is less energy intensive but the required energy is still considerable \parencite{Linow2022Kurzimpuls-Perspektiven2-Emissionen}. For both technologies, the largest fraction of the required energy is the heat required to regenerate the solvent or sorbent, while most of the remainder is used to operate the fans that pump ambient air through the contactors \parencite[203]{NAS2018NegativeAgenda}. Based on the study by \textcite{House2007ElectrochemicalChange}, between 3181 and 5226 kWh of electricity is required to capture one ton of CO\textsubscript{2}\footnote{House assumes that 500 to 800 kJ are required for 1 mol of CO\textsubscript{2}. The result is calculated based on a molar mass of CO\textsubscript{2} of 44.01g and 3600 kJ being equal to 1 kWh}. More recent reports by the \textcite{NAS2018NegativeAgenda} and \textcite{Mulligan2020CarbonShot:States} conclude that about 5 to 10 GJ are required to remove one ton of CO\textsubscript{2}, which converts to a more optimistic 1389 to 2778 kWh/t CO\textsubscript{2}.
However, these results assume that only renewable electricity is used to power the DAC process. Yet, most of the heat demand is currently met using natural gas, resulting in an efficiency penalty of around 30\% due to emissions caused by gas combustion \parencite{Fuss2018NegativeEffects}. If coal is used to power the process, DAC would be completely unviable, since the burned coal releases more CO\textsubscript{2} than is captured \parencite{Erans2022DirectChallenges, NRC2015ClimateSequestration}.\\
The large amount of renewable energy required raises the question of land requirements. For DAC facilities, the \textcite{NAS2018NegativeAgenda} report estimates that about 38000 m\textsuperscript{2} of contactor surface area is required to absorb 1 Mt CO\textsubscript{2}/y. Using the contactor design proposed by \textcite{Holmes2012AnAir}, 10 contactors of size 20x8x200m would be sufficient, with a total land area needed of 0.4 to 1.7 km\textsuperscript{2} \parencite[23]{IEA2022DirectZero}. This is only a fraction of the land used by e.g. AR or the mining operations required for EW.
However, to remove 1 Gt CO\textsubscript{2}/y, about 1.4 TWh of electricity is required, compared to the 1 Twh of renewable energy produced in the US in 2022. At an average production of 170 kWh/m2/y of solar PV\footnote{Average production rate assumed in Germany. The rate varies depending on the geographical region. See also https://photovoltaicsolarenergy.org/solar-panel-yield-per-square-meter/.}, this implies a need of 8000 km\textsuperscript{2} of solar PV to satisfy the entire energy requirements of a 1 Gt/y operation.
Furthermore, a recent study by \textcite{An2022TheSystems} revealed that environmental conditions can impact the efficiency of DAC more than previously expected. They observed that capture rates in hot, humid climates can be increased by a factor of two compared to dry and cold climates. Other issues that must be considered are the rapid degradation of the sorbent material used for SDAC \parencite{Lawler2022WhatTechnology}, the significant water requirement of LDAC of approximately 5 t/t CO\textsubscript{2} \parencite{IEA2022DirectZero} and the somewhat counter-intuitive side effect of local CO\textsubscript{2} depletion, which can lead to lower photosynthesis rates in the direct vicinity of DAC plants \parencite[230]{NAS2018NegativeAgenda}.
\subsection*{Cost}
DAC generally requires substantial investments to build capture facilities and significant ongoing expenditures to cover the energy requirements for capture and regeneration.
For the cost of CO\textsubscript{2} removal, the literature gives a wide range of estimates from 100 to 1000 USD/t. The upper end is based on thermodynamic considerations, while the lower end is proposed by authors closer to industry \parencite{Ishimoto2017PuttingContext, NRC2015ClimateSequestration}.
For LDAC, Carbon Engineering, the first company to commercialize this technology, estimates costs between 168 and 232 USD/t for their plant currently under construction in Texas \parencite{McQueen2021AFuture}. For SDAC, Climeworks estimates that their current costs are around 600 USD/t \parencite[220]{NAS2018NegativeAgenda}.
In the long term, many studies expect costs to fall in the range of 100 - 300 USD/t based on assumed learning effects during ongoing deployment \parencite{Mulligan2020CarbonShot:States, McQueen2021AFuture, NAS2018NegativeAgenda}. \textcite{Lackner2021BuyingCapture} argue that DAC, especially SDAC, is in principle similar to solar PV in that it is a modular technology manufactured in small units at mass scale. \textcite{McQueen2021AFuture} believe that learning rates of 10 to 20\% are realistic. Based on an initial facility of 1 Mt/y capacity at 600 USD/t and a learning rate of 20\%, the 100 USD/t mark could be reached after the deployment of 260 Mt, requiring a total investment of 37.5 billion USD. At a 30\% learning rate, this would decrease to 33 Mt and 5.5 billion USD.\footnote{For a detailed discussion of learning effects, see Appendix B.}
Further cost reduction could also be achieved by developing improved solvents or sorbents with a longer useful life and optimizing the materials used for contactors.
Overall, the \textcite{NAS2018NegativeAgenda} estimates a range of 89 to 877 USD/t.

\section{Bioenergy with Carbon Capture and Storage}
Bioenergy with Carbon Capture and Storage (BECCS) is a CDR strategy that combines bioenergy production with carbon capture and storage (CCS). The concept involves growing biomass such as crops or trees and using it to generate energy through combustion or other conversion methods while capturing the resulting CO\textsubscript{2} emissions. The captured CO\textsubscript{2} is then transported and stored typically in geological formations, resulting in a net-negative effect.
\subsection*{Potential}
Due to the concentration of CO\textsubscript{2} emissions when biomass is used as a feedstock, capturing CO\textsubscript{2} using CCS is easier compared to capturing it from the ambient air using DAC. BECCS can use several different conversion processes such as combustion, biochemical conversion, or thermochemical conversion, depending on the feedstock used. Feedstock can consist of biomass from agricultural waste or municipal solid waste (MSW), forestry, or purpose-grown energy crops \parencite{Dipple2021TheSystems}. Depending on the combination of feedstock and conversion process, total potential and efficiency can vary greatly, with the literature giving estimates ranging from 1 to 85 Gt of CO\textsubscript{2} removal per year \parencite[342]{IPCC2018Global1.5C}.
Based on a model that includes conversion efficiency, energy penalties, and CO\textsubscript{2} capture rates, \textcite{Garcia-Freites2021TheTarget} estimate that 1.1 t CO\textsubscript{2} can be captured per MWh of energy produced with BECCS, suggesting a total global capacity of 790 Mt CO\textsubscript{2}/y if all current bioenergy electricity generation was converted to BECCS.\footnote{The IEA estimated a global bioenergy electricity generation of 718TWh in 2020 (https://www.iea.org/reports/bioenergy-power-generation)}
In the long term, \textcite{Fuss2018NegativeEffects} estimate that 0.5 to 5 Gt CO\textsubscript{2}/y could be sequestered using BECCS by 2050, the lower end using waste biomass only and the upper end requiring significant amounts of dedicated energy crops. \textcite{Pour2018PotentialBECCS} estimate that looking at the growing amount of organic municipal waste, BECCS could remove 2.8 Gt CO\textsubscript{2}/y from MSW alone by 2100. \textcite{Hanssen2020TheStorage} argue that theoretically, 40 Gt CO\textsubscript{2}/y are possible by 2100, but concede that this estimate relies on an unrealistic assumption of land availability.
The IPCC 2018 report projects a realistic range of 3 to 11 Gt CO\textsubscript{2}/y in the long term.
Furthermore, BECCS has the potential to produce excess energy that can be used for other purposes \parencite{Fajardy2017CanEmissions}, such as cross-deployment with DAC or the displacement of some fossil fuels. However, the main motivation for BECCS remains carbon removal, not the generation of electricity \parencite{Klein2014TheREMIND-MAgPIE}.
\subsection*{Limitations}
The greatest limitation of BECCS is, similar to AR, the vast land required to grow the necessary biomass feedstock. The \textcite{IPCC2018Global1.5C} estimates that to sequester 1 t CO\textsubscript{2}/y, between 0.03 and 0.5 ha of land are required. This means that for sequestering 1 Gt CO\textsubscript{2}/y, up to 500 million ha of land would be required, which is a significant portion of the world's total arable land.\footnote{The total global cropland area amounts to about 1.5 billion ha (Source: https://www.worldometers.info/food-agriculture/cropland-by-country/)} At this scale, BECCS would compete with food crops for land, which poses significant ethical concerns regarding the potential impacts on global food security, as rural communities and people in the poorest countries of the world are often most affected by increased food prices. \parencite{Anderson2017TheBECCS, Hanssen2020TheStorage}.\footnote{However, \textcite{Muratori2016GlobalBECCS} note that if bioenergy alone was used to replace fossil fuels without using CCS, the negative impact would be even higher because the use of BECCS reduces carbon prices, which allows for more continued use of fossil fuels, leading to less demand for biomass and therefor also less competition for land and less upward pressure on food prices.}
BECCS also favors the use of fast-growing monoculture plantations of energy crops such as switchgrass or miscanthus \parencite{Fajardy2017CanEmissions}. This decreases land requirements but leads to a loss of biodiversity, disrupts local ecosystems, and can increase the risk of pest and disease outbreaks.
Competition for land also puts the additionality of BECCS into question. If vast amounts of agricultural land are used to grow energy crops, increased deforestation and land-use change might be necessary to compensate for the decreased land available for food production. This, in turn, would cause the release of more CO\textsubscript{2} from the soils and offset at least parts of the sequestration effects of BECCS \parencite{ChathamHouse2020ReachingWork, Muratori2016GlobalBECCS}. It also means that BECCS is in direct competition with other CDR approaches such as AR and soil sequestration.
Furthermore, depending on the regions used to grow energy crops, other side-effects such as a decreased surface albedo rate could reduce the net effect of BECCS \parencite{Hanssen2020TheStorage}.
The growing of large amounts of energy crops also requires high inputs of fertilizers, which are associated with significant GHG emissions, and fresh water. \textcite{Fajardy2017CanEmissions} estimate that to sequester 3.3 Gt CO\textsubscript{2}/y, at least 3.6 billion m\textsuperscript{3} of water and 21 Mt of fertilizers would be required, which would release large amounts of N\textsubscript{2}O.\footnote{The negative impacts of reduced surface albedo and increased fertilizer use were discussed in the chapter on AR.} They also note that emissions along the supply chain can be significant, for example if large amounts of biomass were to be transported from agricultural land to the site of power plants. Therefore, the use of biofuels for transport vehicles and renewable electricity for processing plants would be necessary to reach the maximum potential of BECCS.
If BECCS used only waste biomass, most of these negative effects could be avoided or significantly reduced. However, the amount of CO\textsubscript{2} removed would also be limited to about 1 Gt/y at present, and 2.8 Gt/y by 2100, based on the amount of waste available \parencite{Pour2018PotentialBECCS}.
\subsection*{Cost}
For the cost per ton of CO\textsubscript{2} removed using BECCS, the literature provides a range of estimates of 50 to 250 USD, based on different levels of efficiency and the revenue generated from excess electricity produced. In general, most estimates are below 200 USD \parencite[343]{IPCC2018Global1.5C}. \textcite{Fuss2018NegativeEffects} estimate a narrower range of 100 to 200 USD and argue that lower-end estimates generally assume optimal conversion efficiency and abundantly available biomass. \textcite{Langholtz2020TheUS} focused on the United States and concluded that the realistic range should be between 42 and 137 USD based on a scenario analysis that includes different biomass supply, logistics and power generation approaches.

\section{Summary of Chemical Strategies}
Both chemical processes, DAC and BECCS, are similar in that they both rely on novel technologies that capture carbon dioxide from the air, either directly from the atmosphere or from point sources and then store or sequester it. Both have the potential to remove significant amounts of carbon from the atmosphere and do so faster than the biological or geochemical processes discussed above. In combination, they could remove tens of Gt of CO2 per year and, if geological storage was used, they methods could be highly durable.
However, both methods also come with their own unique set of challenges. For DACs, challenges include high energy requirements and presently high costs. Although its potential is theoretically unlimited, the cost is currently upward of 500 USD/t of CO2 removed, and its viability depends on the use of waste heat, e.g. from geothermal energy. The reduction in costs will depend on the realization of learning effects, investments in further research and development, and policy support. To achieve these cost reductions, it is imperative to begin larger-scale demonstration projects quickly and provide enough public and private funding to support future innovation.
For BECCS, the challenges include the need for significant land and water resources, competition with food production and biodiversity conservation, and possible negative environmental impacts such as soil degradation. BECCS is also largely incompatible with other land-based CDR approaches, which typically focus on more sustainable land use. Therefore, conservative estimates of biomass availability, e.g. from industrial and municipal waste should be used. The deployment of BECCS on a larger scale would require the implementation of a global governance mechanism to oversee land use, ensure sustainable practices, and minimize negative environmental and social impacts. Finally, BECCS also requires a more extensive CCS infrastructure, as carbon sources in bioenergy plants would be more dispersed compared to DAC.
\mychapter{5}{5. Conclusion}
\section{Overall Assessment of CDR}
After reviewing the various approaches to carbon removal and considering the current state of global decarbonization efforts, it can be concluded that CDR approaches demonstrate promising potential for removing carbon from the atmosphere and could therefore play a significant role in achieving the 1.5-degree goal of the Paris Agreement. The reviewed CDR methods differ significantly in their carbon removal mechanisms, resource requirements, and storage durability. Although the realistic utilization of the potential estimated in the literature remains uncertain, it is reasonable to believe that CDR can achieve at least 10 Gt/y of carbon dioxide removal. Furthermore, there is a good chance of reaching 20 Gt/y or more in the long term, which is in line with the requirements set by the National Academy of Sciences (NAS) for 2100. However, there is a high degree of uncertainty regarding the cost of CDR. Although the lower-end cost estimates for all the approaches reviewed fall below the 100 USD mark required to make a CDR approach economically viable, it is important to note that these techniques are still in the early stages of deployment. Even well-understood processes like AR or soil sequestration have only seen limited use as dedicated CDR solutions. Consequently, cost estimates primarily rely on models and lab experiments, resulting in significant variations among studies, and making these estimates ballpark figures at best. The lower-end cost estimates for DAC - one of the most hopeful technologies - are based on assumed learning effects that have not yet been proven. Furthermore, the absence of a universal standard for calculating the cost of carbon removal also means that the estimates presented in individual studies do not necessarily reflect the full costs involved. To create a reliable comparison, a comprehensive lifecycle assessment has to be conducted for each approach.\\
Until now, none of the reviewed approaches can be considered breakthrough solutions. Despite their promising prospects, all proposed CDR approaches suffer from several insufficiently addressed drawbacks. Except for DAC, most methods have substantial land requirements. Although certain approaches such as soil sequestration or EW may be compatible with other land uses, others such as BECCS predominantly occupy land and compete with food crops. Additionally, each approach has potential negative side effects that have not yet been fully understood. Although DAC avoids many of these problems, it is currently not developed sufficiently to serve as a viable breakthrough solution.\\
This means that substantial research and investment are necessary. The uncertainty about the effects of spreading large amounts of mineral and rock material across land and oceans over extended periods requires careful consideration when it comes to approaches like EW and OF. The effectiveness and durability of OF are also uncertain due to the rapid respiration of carbon into the surface ocean and require additional research. Technological advances and significant investment in scale demonstrations are crucial for DAC and, to a lesser extent, BECCS to determine whether learning effects can reduce costs at a sufficient pace.\\
Finally, it is crucial to acknowledge that CDR is not a substitute for emissions reduction. Although CDR may be used to offset emissions that are hard or impossible to eliminate, it cannot replace the need to drastically reduce anthropogenic CO\textsubscript{2} emissions. The scale of CDR required to offset not only 10 to 20 Gt but 60 Gt\footnote{60 Gt/y would be required to remove the approximately 40 Gt/y of anthropogenic CO\textsubscript{2} emissions and the 20 Gt/y required according to the NAS estimate.} or more of CO\textsubscript{2} annually would require an industry similar in size to the current fossil fuel sector.\footnote{Based on the investment, infrastructure, and energy required} The uncertainties surrounding large-scale CDR deployment make relying solely on it an unacceptable gamble.
\\The tables on the following pages provide a condensed overview of potential and cost, as well as of the benefits and risks of the different approaches reviewed in this thesis. For an overview of the carbon removal potential and cost estimates provided in the literature, see Appendix C.
\newpage

\NiceMatrixOptions{cell-space-top-limit=1mm,cell-space-bottom-limit=1mm}

\begin{sidewaystable}
\centering
\caption{Comparison of CDR approaches}
\begin{NiceTabular}{cccc}[hvlines]
\textbf{Technology} & \textbf{Afforestation / Reforestation} & \textbf{Soil Sequestration} & \textbf{Enhanced Mineralization} \\ \hline
\textbf{Potential} & 1.2 - 10 (Avg.: 5.8) & 1.2 - 3.57 (Avg.: 2.4) & 2.5 - 10 (Avg.: 4.9) \\ \hline
\textbf{\begin{tabular}[c]{@{}l@{}}Cost\\ (USD/t CO\textsubscript{2})\end{tabular}} & 1 - 494 (Avg.: 97) & 10 - 100 (Avg.: 45) & 24 - 600 (Avg.: 225) \\ \hline
\textbf{CAPEX} & Low - Medium & Medium & Medium - High \\ \hline
\textbf{OPEX} & Low & Low & High \\ \hline
\textbf{Cost drivers} & \begin{tabular}[c]{@{}l@{}}Land required,\\ management cost\end{tabular} & \begin{tabular}[c]{@{}l@{}}Cost of adapting to new\\ land management techniques\end{tabular} & \begin{tabular}[c]{@{}l@{}}Construction of infra-\\ structure,processing and\\ transportation\end{tabular} \\ \hline
\textbf{\begin{tabular}[c]{@{}l@{}}Resource\\ requirements\end{tabular}} & Land, water & None (doesn't prevent land use) & Rock, energy \\ \hline
\textbf{Durability} & Medium & Medium & Highest \\ \hline
\textbf{\begin{tabular}[c]{@{}l@{}}Risks to\\ durability\end{tabular}} & Fires, pests & \begin{tabular}[c]{@{}l@{}}None, but requires\\ continuous and\\ permanent usage\end{tabular} & None \\ \hline
\textbf{Additionality} & \begin{tabular}[c]{@{}l@{}}Medium, converting farmland\\ back into forests may\\ result in forest removal in\\ other locations\end{tabular} & High & High \\ \hline
\textbf{Co-Benefits} & Can be used with agroforestry & \begin{tabular}[c]{@{}l@{}}Improved soil quality,\\ reduced land erosion\end{tabular} & \begin{tabular}[c]{@{}l@{}}Can improve soil fertility,\\ reduce ocean acidity\end{tabular} \\ \hline
\textbf{\begin{tabular}[c]{@{}l@{}}Negative\\ side effects\end{tabular}} & Possible competition for land & none & \begin{tabular}[c]{@{}l@{}}Possible release of toxic\\ metals to the food chain\end{tabular} \\ \hline
\textbf{\begin{tabular}[c]{@{}l@{}}Difficulty of\\ verification\end{tabular}} & Medium, land-based & Medium, land-based & \begin{tabular}[c]{@{}l@{}}Medium, land-based\\ or activity-based\end{tabular} \\ \hline
\end{NiceTabular}
\end{sidewaystable}

\begin{sidewaystable}
\centering
\caption{Comparison of CDR approaches (cont.)}
\begin{NiceTabular}{cccc}[hvlines]
\hline
\textbf{Technology} & \textbf{Ocean Fertilization} & \textbf{DAC} & \textbf{BECCS} \\ \hline
\textbf{Potential} & 0.3 - 5 (Avg.: 2) & 1.2 - 15 (Avg.: 5.8) & 0.3 - 12 (Avg.: 5.85) \\ \hline
\textbf{\begin{tabular}[c]{@{}l@{}}Cost\\ (USD/t CO\textsubscript{2})\end{tabular}} & 20 - 457 (Avg.: 115) & 60 - 1000 (Avg.: 352)\tabularnote{The cost for DAC and BECCS does not include the cost of storage or sequestration, which is considered to be around 20 USD/t.} & 42 - 300 (Avg.: 147)\tabularnote{The cost for DAC and BECCS does not include the cost of storage or sequestration, which is considered to be around 20 USD/t.}\\ \hline
\textbf{CAPEX} & Low - Medium & High & Medium - High \\ \hline
\textbf{OPEX} & Medium & High & Medium \\ \hline
\textbf{Cost drivers} & \begin{tabular}[c]{@{}l@{}}Cost of mining and spreading\\ nutrients\end{tabular} & \begin{tabular}[c]{@{}l@{}}Construction of facilities,\\ energy requirements\end{tabular} & \begin{tabular}[c]{@{}l@{}}Land required, fertilization,\\ processing, construction of\\ bioenergy plants\end{tabular} \\ \hline
\textbf{\begin{tabular}[c]{@{}l@{}}Resource\\ requirements\end{tabular}} & Rock & Vast amounts of energy & Land, water, fertilizer \\ \hline
\textbf{Durability} & Questionable & \begin{tabular}[c]{@{}l@{}}Depends on storage approach\end{tabular} & Depends on storage approach \\ \hline
\textbf{\begin{tabular}[c]{@{}l@{}}Risks to\\ durability\end{tabular}} & \begin{tabular}[c]{@{}l@{}}None if sequesterd on ocean\\ floor, but most CO\textsubscript{2} captured\\ is respired back to surface quickly\end{tabular} & - & - \\ \hline
\textbf{Additionality} & \begin{tabular}[c]{@{}l@{}}Questionable, due to possible\\ nutrient robbing\end{tabular} & Highest & \begin{tabular}[c]{@{}l@{}}Medium, possibly high\\ competition for land\end{tabular} \\ \hline
\textbf{Co-Benefits} & \begin{tabular}[c]{@{}l@{}}Phytoplankton can increase\\ oxygen content of oceans\end{tabular} & \begin{tabular}[c]{@{}l@{}}Can be used to clear air\\ from pollution or draw\\ water from the ambient air\end{tabular} & \begin{tabular}[c]{@{}l@{}}Electricity production,\\ displacement of fossil fuels\end{tabular} \\ \hline
\textbf{\begin{tabular}[c]{@{}l@{}}Negative\\ side effects\end{tabular}} & \begin{tabular}[c]{@{}l@{}}Nutrient robbing, acidification\\ of deep ocean\end{tabular} & \begin{tabular}[c]{@{}l@{}}CO\textsubscript{2} depletion of\\ local ecosystems\end{tabular} & \begin{tabular}[c]{@{}l@{}}Risk of disease for mono-\\ cultures, threat for food security\end{tabular} \\ \hline
\textbf{\begin{tabular}[c]{@{}l@{}}Difficulty of\\ verification\end{tabular}} & \begin{tabular}[c]{@{}l@{}}High, requires \\ measuring of carbon content\\ of deep ocean\end{tabular} & Low & Low \\ \hline
\end{NiceTabular}
\end{sidewaystable}
\FloatBarrier

\newpage
\section{Implementation Recommendations}
Based on the review of carbon dioxide removal strategies, several recommendations can be made for their implementation.\\
\\(1) Implement the cheapest and most available options quickly\\
Soil carbon sequestration is the only option that does not require significant new infrastructure or large amounts of energy. It can be easily introduced and can be cost-positive for farmers in many cases. Implementation relies largely on policy support for education and outreach. Also, BECCS should be implemented to use available biomass waste that does not require additional land use. In combination, soil sequestration and waste-based BECCS could probably provide CO\textsubscript{2} removal in the range of multiple hundred Mt/y until 2030.\\
\\(2) Invest in the deployment of DAC\\
DAC is one of the most promising options, since it does not require significant land and can be located anywhere. It also comes with minimal side effects compared to other approaches. However, significant development and practical testing are necessary to understand and realize learning effects that could reduce costs and increase efficiency, and the required funding of 5 to 35 billion USD cannot be provided by private entities alone. Instead, governments and intergovernmental entities should allocate sufficient funds to support research and pilot projects and develop DAC to economic viability. If successful, DAC could provide CO\textsubscript{2} removal in the Gt/y range by 2050 and become the largest contributor to global carbon removal efforts.\\
\\(3) Conduct more research on alternative technologies and innovations\\
While much hope can be placed in the aforementioned strategies, it is important to continue researching other CDR technologies as well. This will enable the development of a diverse portfolio of approaches and allow flexibility in implementation, for example if the cost of DAC cannot be reduced to the required level. Approaches that may prove worthwhile include ocean fertilization using iron nanoparticles or other more recent "moonshot" technologies such as direct ocean capture or the development of genetically engineered trees that sequester increased amounts of CO\textsubscript{2}. Although they may seem hardly feasible at the moment, these innovative approaches should not be ignored, as further research may lead to unexpected breakthroughs.
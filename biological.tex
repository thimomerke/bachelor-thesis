\mychapter{2}{2. Biological Strategies}
\section{Afforestation and Reforestation}
Trees store CO\textsubscript{2} by converting it into plant biomass by photosynthesis. \textcite{Crowther2015MappingScale} estimate that there are approximately three trillion trees worldwide, and forests make up 38\% of all habitable land area \parencite{RitchieHowForested}. Despite significant deforestation in the past, forests sequester 7.6 Gt of CO\textsubscript{2} annually, removing nearly 20\% of anthropogenic carbon emissions from the atmosphere \parencite{Harris2021GlobalFluxes}.
\subsection*{Potential}
Afforestation refers to the cultivation of forests in areas that were recently deforested, whereas reforestation occurs on lands that did not grow forests in the recent past.\footnote{'Recent past' is typically defined as the last 30 to 50 years.} The total potential for carbon sequestration through afforestation and reforestation (AR) depends on three factors: (1) the amount of land available, (2) the amount of carbon that trees can sequester, and (3) the environmental climatic conditions.
The IPCC 2000 report estimates that trees in boreal regions sequester up to 1.2 t CO\textsubscript{2}/ha/y, while trees in temperate regions can store 4.5 t CO\textsubscript{2}/ha/y, and trees in tropical rainforests can store up to 8 t CO\textsubscript{2}/ha/y. These differences are explained partially through the different lengths of the growing season, which only lasts a few months in boreal climates but spans almost the entire year in tropical regions \parencite[172-173]{Watson2000LandForestry}. Other reports assume higher storage potential, with the \textcite{NAS2018NegativeAgenda} claiming an upper bound of 30 t CO\textsubscript{2}/ha/y. Overall, the IPCC report estimated that an additional 87 Gt could be sequestered using AR between 1995 and 2050, resulting in an annual uptake of 1.58 Gt CO\textsubscript{2}, 75\% of which would be stored in tropical rainforests. More recent reports, such as \textcite{Fuss2018NegativeEffects} and \textcite{Griscom2017NaturalSolutions} use higher theoretical estimates of up to 10 Gt/y.\\
In addition to carbon removal, AR can bring benefits such as increased biodiversity, reduced land erosion and desertification in dry areas, and protection of watersheds \parencite[177]{Watson2000LandForestry}.
\subsection*{Limitations}
Carbon removal through AR comes with extensive land requirements. At a storage rate of 10 t CO\textsubscript{2}/ha/y, sequestering 1 Gt CO\textsubscript{2}/y would require 100 million hectares of additional forests.\footnote{For comparison, Germany has a total surface area of about 36 million hectares.} Cultivating ecologically sustainable forests can also prove to be difficult. Creating vast monocultures of a single species of trees or turning previous grasslands into forests can present unforeseen dangers and negatively impact ecosystem diversity. This also raises the question of durability. Compared to other, more durable carbon removal strategies, forests are vulnerable to wildfires, pests, and other disturbances that can cause rapid loss of stored carbon. Therefore, using forests as carbon storage requires ongoing fire and pest management \parencite[216]{Watson2000LandForestry}. Depending on the region, other factors, such as surface albedo and evapotranspiration, can also play a role and significantly reduce the effectiveness of carbon storage. Forests generally have lower albedo rates than grasslands, meaning that they reflect a lower portion of the incoming sunlight and therefore trap a larger amount of heat. This effect is especially strong in high-latitude forests. In comparison, evapotranspiration occurs mainly in tropical rainforests, where large amounts of water evaporate and, while cooling the surface, can act as a potent greenhouse gas and cause atmospheric warming \parencite[46]{NRC2015ClimateSequestration}. Finally, the net additionally of AR also depends on what the land was previously used for. If farmland is converted to forests, then it is likely that, considering the increasing food demand worldwide, forests elsewhere will be cut to ensure sufficient supply. This would significantly reduce or fully eliminate the net effect of AR. The same applies if forests are planted on lands that would naturally have regrown forests anyway \parencite[128]{Gates2021HowDisaster}.\\
Carbon storage by trees is also limited by their growth rate and natural lifecycle. Carbon uptake is generally the highest during the growth phase, which can be between 5 and 150 years depending on the species, but slows down afterward \parencite[268]{Watson2000LandForestry}. The highest uptake rate is usually reached after 30 to 40 years in most regions, depending on environmental factors \parencite[40]{NRC2015ClimateSequestration}. The \textcite[215]{Watson2000LandForestry} proposes using non-organic nitrogen fertilizers to increase growth rates and carbon uptake. However, a study by \textcite{Crutzen2008AtmosphericPhysics} has shown that nitrogen fertilizer is partially converted to nitrous oxide, whose greenhouse potential is 300 times greater than CO\textsubscript{2}. The assumed amount of nitrogen fertilizer required ranges from 24 kt/y \parencite{Rau2013DirectProduction} to 500 kt/y \parencite{Dipple2021TheSystems} for 1 Gt CO\textsubscript{2}/y stored and in the worst case scenario could create more CO\textsubscript{2}e\footnote{CO\textsubscript{2} equivalents, a measure that expresses the global warming potential of other GHGs relative to CO\textsubscript{2}} than the tree growth removed \parencite[40]{NRC2015ClimateSequestration}.\\Finally, climate change itself is strongly affecting forest ecosystems. While increased levels of atmospheric CO\textsubscript{2} and warmer temperatures can increase tree growth, an increase in the frequency of droughts, wildfires, and other extreme weather events disturbs forest growth and negatively affects plant and animal life \parencite{EPA2015ClimateForests}, making AR efforts more difficult.
The verification of the effects of AR is also an issue. Although it is possible to measure forest growth using a land-based approach, the accuracy of such measurements can be limited, and the measuring capabilities can vary by country \parencite[217]{Watson2000LandForestry}.
\subsection*{Cost}
The cost of AR is largely dependent on the cost of land, the terrain and the type of trees that are planted. According to \textcite{Gorte2009U.S.Sequestration}, the cost of reforestation after severe wildfires in the United States in 2002 ranged between 200 and 2000 USD/acre\footnote{1 acre = 0.405 hectares}, and the average cost was 523 USD/acre in 2007. At a storage rate of 10 t CO\textsubscript{2}/ha/y, this suggests costs between 49 USD/t CO\textsubscript{2}  and  494 USD/t CO\textsubscript{2}, and an average of 129 USD/t CO\textsubscript{2}.\\Other studies resulted in slightly lower cost estimates, with a lower limit of 1 USD/t CO\textsubscript{2}, a cost of 7.5 to 22 USD/t CO\textsubscript{2} for sequestration of 1 Gt CO\textsubscript{2}/y, and half of the total sequestration potential being available below 55 USD/t CO\textsubscript{2}. However, it remains questionable whether these estimates adequately account for the possible risks of fires, pests, or other disturbances that could lead to the release of sequestered carbon \parencite[41-42]{NRC2015ClimateSequestration}.\\Additional cost reductions could be achieved by using and systematizing agroforestry, an approach that integrates trees into crop pastures and comes with numerous co-benefits that can offset the cost of tree cultivation, and is useful especially in economically disadvantaged regions and dry climates \parencite{Reij2014ImprovingAchievable}.

\section{Soil Sequestration}
According to the 2000 IPCC report, soil cultivation can result in the release of more than half of the carbon stored in it into the atmosphere. The \textcite{NRC2015ClimateSequestration} estimates that in the last ten thousand years agricultural activities have caused a net release of up to 840 Gt of CO2. Despite this loss, soils are still an important carbon storage, containing up to 10\% of the global carbon stocks.
\subsection*{Potential}
Using a number of improved land management techniques, such as conservation tillage practices, planting cover crops, and optimizing crop types and fertilization practices, the loss from agricultural use can be partially reversed and the amount of organic matter in soils, more than half of which is typically stored not above ground, but in roots, dead leaves, and wood litter, can be increased \parencite[192]{Watson2000LandForestry, Ontl2012SoilStorage}. This increased sequestration serves both economic and environmental purposes, as it increases the amount of carbon stored while potentially increasing crop yields and promoting soil health. \parencite{Dipple2021TheSystems}\\
Improved land management practices are well-understood, have been in use in the United States since the 1950s, and have since seen increased adoption in Asia and parts of South America \parencite[202]{Watson2000LandForestry}. Compared to forest-based carbon removal approaches, carbon sequestered in soils is less vulnerable to fires and other diseases, making it a more reliable form of carbon storage \parencite{Dipple2021TheSystems}. The theoretical potential of soil sequestration for carbon storage is determined by the historical stock before the beginning of agricultural use, which caused much of the degradation of the ecosystem and the resulting loss of soil carbon \parencite{NAS2018NegativeAgenda}.
\textcite{Dipple2021TheSystems} estimate that soils are currently a net sink of CO\textsubscript{2}, with an annual uptake of 1 to 2 Gt CO\textsubscript{2}, mostly caused by the increased CO\textsubscript{2} concentration in the atmosphere itself.\footnote{This effect is also referred to as "CO\textsubscript{2} fertilization".}
However, the use of optimized land management practices can significantly increase carbon uptake. Conservation tillage is one of the most effective techniques. It involves a reduction or complete elimination of tilling to prevent soil disturbance, while also leaving at least 30\% of the residual crop biomass from agricultural use in the field. Originally developed to reduce soil erosion, studies found that by minimizing soil disturbance and maintaining crop residues, organic matter aeration and decomposition is reduced, leading to increased soil organic carbon (SOC) and prolonged carbon storage. Studies estimate that an average of 1.2 t of CO\textsubscript{2} could be stored per hectare of farmland annually through conservation tillage in approximately 60\% of the total arable land. With a total cropland area of approximately 1.5 billion hectares worldwide, conservation tillage has the potential to sequester a total of 0.33 to 4.3 Gt CO\textsubscript{2}/y\footnote{The 2000 IPCC report (p. 202) estimates carbon storage of 0.1 to 1.3 t C/ha. Based on molecular mass, this is equal to 0.37 to 4.9 t CO\textsubscript{2}/ha; applying that to 60\% of 1.5 billion ha worldwide yields these results.} \parencite{Watson2000LandForestry, NRC2015ClimateSequestration}.
Another promising method of soil carbon sequestration is the use of cover crops, which are planted between profit crop seasons to protect and improve soil quality. These crops provide continuous soil cover, which reduces erosion, promote biological activity in the soil, and supply organic matter to the soil through roots and biomass. They also increase nitrogen content and serve as natural fertilizers \parencite{Ontl2012SoilStorage,Dipple2021TheSystems}.
In total, studies estimate that soils could sequester between 3.3 and 5.2 Gt CO\textsubscript{2}/y globally \parencite{NRC2015ClimateSequestration, Bossio2020TheSolutions, Dipple2021TheSystems}.

\subsection*{Limitations}
The two biggest limitations facing soil sequestration through improved land management are the limited potential for carbon absorption per hectare of agricultural land and the need for permanent maintenance. Naturally, the amount of carbon that soil can absorb is limited. \textcite{Smith2016SoilTechnologies} estimates that soil saturation will occur after 10 to 100 years, depending on the type of soil and environmental conditions, and that it will be slower in colder climates. The 2000 IPCC report estimates that on average, soil sequestration reaches its highest annual potential within the first 5 to 20 years after adopting conservation tillage and then declines to near-zero after 50 years. Consequently, if started globally today, soil sequestration would not be available as a sink for the second half of the 21st century.  This problem is increased by the fact that, while uptake is limited, it is also easily reversible, meaning that a return to previous, standard land use and management techniques would release the carbon stored previously. Thus, improved land management techniques have to be used indefinitely and the costs that could be associated with them have to be paid continuously. \parencite{Smith2016SoilTechnologies}.
Therefore, the use of optimized land management techniques faces not only technical but also socioeconomic problems \parencite{Dipple2021TheSystems}. \textcite{Zelikova2020LeadingAgriculture} argue that to increase the incentive for farmers to implement better land management, demonstration projects should be conducted and the support for farmers should be increased.

\subsection*{Cost}
Due to the wide range of different agricultural processes involved and the difficulty in measuring their exact costs, the economic viability of soil carbon sequestration techniques such as conservation tillage and cover cropping is difficult to assess. \textcite{Bossio2020TheSolutions} expect that half of the total soil sequestration potential is available below 100 USD/t CO\textsubscript{2}, while one-fourth is available below 10 USD/t CO\textsubscript{2}, based on the available marginal abatement cost curves. A study by \textcite{McKinsey2009PathwaysEconomy} reaches a similar conclusion. However, these estimates include sequestration not only on agricultural land but also in forests. \textcite{Bossio2020TheSolutions} also note that, based on the co-benefits of improved soil health and reduced inputs, some soil sequestration can be cost-effective, i.e. have negative costs. Yet, \textcite{Zoebisch2022SoilMoonshot} remark that these co-benefits can take some years to become effective, while the initial costs of implementing new practices can be high.
\textcite{Chambers2016SoilInitiative} used the subsidies paid to farmers in the US for the adoption of such practices as a reference point and found costs to be between 3.44 and 11.34 USD/t CO\textsubscript{2} sequestered. However, this estimate is based on a small-scale study, only includes the sum paid to farmers by the government, and may not be representative of the costs associated with the broader implementation of these measures.

\section{Summary of Biological Strategies}
Overall, since biological carbon removal technologies are based on the usage and enhancement of natural processes that have occurred for millions of years, they are well understood and have a theoretically large carbon removal potential in the range of 5 to 10 Gt CO\textsubscript{2}/y. Their implementation is very straightforward, and mostly an issue of deploying them on a large-enough scale. The biggest advantage of biological carbon removal is that most of the costs are incurred during the initial implementation, i.e. for planting trees and adapting to new land management processes, while the operating costs, under ideal circumstances, could be relatively low. However, there is relatively little experience with the dedicated application of these methods for carbon removal purposes. While studies generally point to the costs being below 100 USD/t CO\textsubscript{2}, which would make biological strategies economically viable, the vast land requirements, the limited amount of carbon that biomass and soil can sequester, and the risk of natural disasters such as wildfires and pests impacting the permanence of carbon storage cast doubt on whether these strategies can be used for long-term, large-scale carbon removal operations.\\Additionally, it should be noted that with an ever-increasing world population, the demand for food will continue to rise, thus potentially increasing competition for land or forcing farmers to choose between using long-term, sustainable farming techniques (i.e. those that also allow increased carbon sequestration) and maximizing shorter-term yields.
\begin{abstract}
This thesis examines the economic and environmental viability of carbon removal strategies as a means of mitigating climate change. It provides an overview of the current state of global greenhouse gas emissions and carbon removal efforts. It then explores various carbon removal strategies and analyzes economic and environmental factors affecting their feasibility, as well as their potential co-benefits and adverse side effects. The analysis finds that while some carbon removal strategies are more cost-effective at present, all have the potential to play a role in mitigating climate change. However, careful consideration must be given to the potential side effects of these strategies, particularly in terms of land use change and ecosystem disruption. Based on the results, the thesis also provides recommendations for implementing carbon removal strategies. Ultimately, carbon removal cannot replace the urgent need for reductions in CO\textsubscript{2} emissions, but it can complement such efforts.
\end{abstract}
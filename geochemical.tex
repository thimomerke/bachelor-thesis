\mychapter{3}{Geochemical Strategies}
\section{Enhanced Mineralization}
Carbon mineralization is a chemical process in which alkaline materials such as magnesium and calcium silicates react with water and carbon dioxide to produce stable, solid carbonates. Similarly to the aforementioned biological processes, it has occurred naturally for millennia, albeit at a very slow rate.
\subsection*{Potential}
Mineralization strategies include enhanced weathering (EW), which involves the spreading of finely-ground alkaline minerals over large land surfaces to naturally react with atmospheric CO\textsubscript{2}, in-situ processes, for which concentrated CO\textsubscript{2} streams are injected into geological formations containing appropriate minerals, and ex-situ processes, which involve the extraction and processing of alkaline materials to enhance their reactivity with CO\textsubscript{2}. All of these processes fixate carbon dioxide in stable carbonate minerals, which can then be stored for hundreds or thousands of years and avoid the risk of leakage. \\
Sources of alkalinity are typically minerals such as peridotites and serpentinites, which are abundandly available \parencite{Lackner1997ProgressSubstrates}, and mineral-heavy industrial byproducts such as waste materials produced during resource extraction, construction, and manufacturing, which are highly reactive and available at little cost \parencite{NationalResearchCouncil2015ClimateSequestration}.
Since 38\% of all global land area are cropland or pastures, which are ideal for spreading rock dust \parencite{Almaraz2022MethodsSettings}, EW theoretically has a huge potential for carbon sequestration. \textcite{Beerling2018FarmingSecurity} estimate that if two-thirds of the most productive cropland worldwide (a total area of 900 million hectares) were treated with rock dust at a rate of 20-30 t/ha/y, a sequestration potential of 0.5 - 4Gt CO\textsubscript{2}/y could be achieved. They also argue that treating cropland with basalt has been shown to have significant co-benefits, including increased soil fertility and crop yields, and an increased amount of organic material in soils. Combined with the previously mentioned improved soil management techniques, this could make EW an attractive strategy from both a carbon removal and agricultural productivity perspective. Another study by \textcite{Goll2021PotentialRock} proposes spreading a single dose of 5t of basalt dust per hectare on 5500 million hectares of unpopulated land, and based on a land-surface model that uses experimental weathering data, estimates a sequestration potential of 2.5Gt CO\textsubscript{2}/y over the next 50 years.\\
For ex-situ approaches, \textcite{Lackner1997ProgressSubstrates} propose an exothermic chemical process that uses a reactor vessel and concentrated CO\textsubscript{2} atmosphere to carbonize alkaline minerals under pressure without requiring large additional energy inputs. For in-situ processes, proof-of-concepts that inject carbon-rich liquid in underground basalt formations are currently underway, with one experimental plant in Iceland sequestering about 12Kt CO\textsubscript{2}/y. In optimistic estimates, both of these processes could also sequester multiple gigatons of CO\textsubscript{2} per year if deployed on a large scale \parencite{Dipple2021TheSystems}.
\subsection*{Limitations}
The primary issue of enhanced mineralization is that it requires a constant supply of fresh rock and large amounts of energy to mine, grind and transport the rock dust. For EW, \textcite{Schuiling2006EnhancedCo2} estimate that about 2t of olivine rock dust are required per tonne of CO\textsubscript{2} removed, based on a calculation of the amount of rock required per square meter to absorb the entire atmospheric CO\textsubscript{2}, which was estimated at 5kg/m\textsuperscript{2} of surface area.\footnote{The full calculation is as follows: Schuiling et al. estimate that a rock dust layer of 0.4cm would be required on all global land area to remove the entire CO\textsubscript{2} content from the atmosphere. This equals 0.004m\textsuperscript{3} of rock dust per m\textsuperscript{2} of surface area, which at a density of about 2500kg/m\textsuperscript{3} results in a requirement of 10kg of olivine dust.} If basalt is used, the amount of rock dust required is even greater, with \textcite{Beerling2018FarmingSecurity} estimating that up 6.75t of basalt per tonne of CO\textsubscript{2} could be needed. On a global scale, this would require between 9 and up to 27 Gt of basalt to remove a maximum of 4Gt of CO\textsubscript{2} annually. The study by \textcite{Goll2021PotentialRock} arrived at a more optimistic estimate of 5.5Gt of rock dust needed to remove 2.5Gt CO\textsubscript{2}/y, albeit on unpopulated land.\footnote{For comparison, about 8Gt of coal and 2.5Gt of iron ore are mined globally per year.} Extracting, processing and transporting such large amounts of rock dust would require significant energy and resource inputs, which would not only be costly but also offset between 10 - 30\% of the carbon sequestered. Large-scale mining operations also necessarily disrupt ecosystems and have negative ecological impacts \parencite{Beerling2018FarmingSecurity}. \textcite{Almaraz2022MethodsSettings} note that there is little experience with the ecological impact of rock dust application, and warn that heavy metal contents in the rock material could accumulate in the soil and then be introduced into the food chain.\\The second issue is the land requirement: Due to the relatively slow weathering, it could take years to decades to fully mineralize the rock dust. Weathering speed also heavily depends on environmental circumstances. While acidic rain with pH values around 4, which occurs e.g. in Europe and North America, can mineralize the proposed rock dust layer within 30 years, an pH increase of just 0.3 units could double the required time \parencite{Schuiling2006EnhancedCo2}. This means that a large percentage of the total land area would have to be treated, and while rock dust application does usually not compete with other land usages such as agriculture and forestry, it would still be a significant logistical challenge \parencite{Dipple2021TheSystems}.\\Another issue is the question of how the mineral dust would be spread. For farmland application, small-scale experiments in Germany by the \textcite{CarbonDrawdownInitiative2022HowExperiments} have shown that using existing agricultural spreaders, such as manure spreaders or fertilizer spreaders, is the only method that can effectively and efficiently distribute rock dust, with the spreading of 40t on one hectare of farmland taking a little more than 30 minutes. However, this process requires the use of larger-grained rock coarse instead of super-fine dust, which results in a less homogenous rock layer and makes monitoring results more difficult. The Goll proposal on the other hand includes the usage of spreader aircraft for distribution, which would require significant resources and infrastructure to operate, further adding to the costs of implementation.\\Lastly, it should be considered that for ex-situ processes, the study by \textcite{Lackner1997ProgressSubstrates} admits that to speed up the mineralization to complete within hours instead of days, high-pressure and high-temperature reactors are required, which could result in significant energy requirements. Also, both in-situ and ex-situ reaction processes (excluding EW) are strictly speaking not carbon-removal but carbon-storage processes, since they require the usage of pre-concentrated CO\textsubscript{2}, e.g. from DAC or point sources in industrial plants.
\subsection*{Cost}
The costs associated with the implementation of enhanced mineralization are somewhat difficult to calculate, considering the number of steps involved in the process, from mining to spreading (or mineralizing) the rock dust. While \textcite{Almaraz2022MethodsSettings} argue that EW can be economically viable for farmers due to the previously mentioned co-benefits, they don't provide specific numbers. These benefits also don't apply to other in-situ or ex-situ mineralization processes. \\
Overall, the biggest share of the costs would come from the extraction, processing, and transportation of large amounts of rock dust. Studies by \textcite{Beerling2018FarmingSecurity} and \textcite{Fuss2018NegativeEffects} provide wide-range estimates for EW, with a minimum of 50 US\$/t and a maximum of 578 US\$/t of CO\textsubscript{2} removed, mostly based on the cost for grinding the basalt rock and transporting the dust to its destination. The study by \textcite{Goll2021PotentialRock} comes to a similar conclusion, stating that 0.2Gt CO\textsubscript{2}/y should be available below 100 US\$/t CO\textsubscript{2}, and up to 2.5Gt/y below 500 US\$/t CO\textsubscript{2}, based on the logistics cost for application in uninhabited areas.
The \textcite{Lackner1997ProgressSubstrates} study estimates costs for raw materials to be between 40 and 100 US\$/t CO\textsubscript{2} removed for ex-situ mineralization, and argue that they could be reduced further once large-scale production begins. However, this estimate does not include the costs for energy required, nor the cost of building and maintaining the required infrastructure and industrial-scale processing plants. If energy requirements were included, \textcite{Lawler2021CarbonMineralization} argue that costs could be as high as 600 US\$/t CO\textsubscript{2} removed for ex-situ mineralization, but do not provide a detailed basis for this calculation. They also argue that ex-situ mineralization could be cost-effective, albeit at a limited scale, if the mineralization byproducts were sold to e.g. the paper industry as a replacement for lime. For in-situ mineralization, even less data is available. CarbFix, the company currently running one of the largest in-situ mineralization projects, estimated their costs to be around 25 US\$/t CO\textsubscript{2} sequestered \parencite{Lawler2021CarbonMineralization}. While this relatively low estimate is promising, it is important to note that it only accounts for the costs of carbon sequestration, while the costs for capturing it using e.g. DAC technology are not included.
%\section{(Alkalinity Enhancement)}
\section{Ocean Fertilization}
During its growth, phytoplankton absorbs solved carbon dioxide from the surface ocean through photosynthesis. When it dies, it falls to the ocean floor, taking the CO\textsubscript{2} it absorbed with it. There, it is consumed by fish and other animals, which respire and convert the organic carbon back into CO\textsubscript{2}. This cycle is called the biological carbon pump (BCP). The idea of ocean fertilization is to stimulate phytoplankton growth and strengthen the BCP to sequester more carbon in the oceans.
\subsection*{Potential}
In most oceans, the growth of phytoplankton is limited by the availability of nutrients such as iron, nitrogen, and phosphorus. The first step of ocean fertilization (OF) is to supply these nutrients, either by adding the micronutrient iron (Ocean Iron Fertilization, OIF) or the macronutrients nitrate and phosphate (Ocean Macronutrient Fertilization, OMF) \parencite[77]{NationalAcademiesofSciences2022ASequestration}.
OIF was first popularized by John Martin ("Give me half a tanker of iron, and I will give you the next ice age.") and is proposed for regions where macronutrients are high, but phytoplankton content is low (i.e. iron where iron is the limiting factor). These high-nutrient, low-chlorophyll (HNLC) regions are mainly in the Southern Ocean, the Equatorial Pacific, and parts of the North Atlantic. OMF on the other hand would be applied to low-nutrient, low-chlorophyll (LNLC) regions \parencite{Chisholm2001Dis-CreditingFertilization}.
In the second step, the stimulated phytoplankton growth takes up CO\textsubscript{2} from the surface ocean, converting it into organic matter through photosynthesis. Once this organic matter falls to the ocean floor, its carbon content are incorporated into sediment and sequestered for hundreds of years or even millennia. \parencite{Fuss2018NegativeEffects, S.F.Jones2014TheNourishment}\\
OIF has shown promising results in increasing phytoplankton growth and the uptake of CO\textsubscript{2} in HNLC. It is also favored due to the low ratio of iron required to the carbon sequestered of approximately 1:1000 \parencite[99]{NationalAcademiesofSciences2022ASequestration}.
The study by \textcite{Hauck2016IronExperiment} found the global sequestration potential of OIF to be 113Gt CO\textsubscript{2} per century based on an ecosystem model that includes nitrate, silicic acid, and iron, but does not account for potential sea-air backflow of carbon. At an average rate of 1.1Gt CO\textsubscript{2}/y, they require an input of 2.3Mt Fe/y into the oceans. They also propose coupling  OIF with EW, since the olivine minerals used in EW release iron that can be used for OIF.
\textcite{Keller2014PotentialScenario} give a more optimistic estimate of up to 15Gt CO\textsubscript{2}/y, based on a scenario in which the entire southern ocean would be fertilized continuously, but admit that this number would quickly decrease to a maximum 5Gt CO\textsubscript{2}/y as the large reservoir of available macronutrients is used up.
Overall, the \textcite[87]{NationalAcademiesofSciences2022ASequestration} found a realistic potential for OIF in HNLC regions of 1Gt CO\textsubscript{2}/y.\\
For OMF, \textcite{Harrison2017GlobalFertilization} estimates a one-off potential of 3.6Gt CO\textsubscript{2}, with about 0.7Gt CO\textsubscript{2}/y afterwards if only nitrate is added. Their estimate is based on the quantity of phosphate (which would be the limiting factor) available in the oceans, and the amount of nitrogen needed to "use up" that phosphate, combined with the expected efficiency in boosting phytoplankton growth. In a second scenario, they consider the addition of both nitrate and phosphate, which increases the annual potential to 1.5Gt CO\textsubscript{2}, with only the amount of mined phosphate added being the limiting factor.
\subsection*{Limitations}
OF is limited to specific regions of the oceans which present ideal conditions for fertilization (i.e. HNLC for OIF and LNLC for OMF). This largely limits OIF to the Southern Ocean encircling Antarctica. For OMF, the viable oceanic regions are easier to access since large parts of the global oceans are LNLC, however, the amount of material (nitrate and phosphate) required is huge in comparison, based on the Redfield ratio\footnote{The Redfield ratio of 106:16:1 describes the ratio of carbon to nitrogen to phosphorus (C:N:P) found in phytoplankton.} and calculations by \textcite{S.F.Jones2014TheNourishment} and \textcite{Harrison2017GlobalFertilization}, and in addition to the required logistics would make up a significant share of the total phosphate produced worldwide.\\
The biggest concern however is that only a small fraction of the organic carbon is transferred to the deep ocean and buried. When phytoplankton dies and starts sinking to the ground, most of it is consumed by zooplankton, microbes, and other animals within 100 meters of the photic zone. Their respiration releases the carbon, which quickly returns to the surface ocean and eventually the atmosphere \parencite[82, 84]{NationalAcademiesofSciences2022ASequestration}.
\textcite{Zeebe2005FeasibilityLevels} estimates a sedimentation rate of only 10 - 25\% based on ocean carbon cycle models. This means that of 1Gt CO\textsubscript{2} absorbed by phytoplankton, only 0.1 - 0.25Gt are incorporated into the bottom sediment while the rest is returned to the surface quickly. Furthermore, \textcite{Harrison2013AOcean} estimates that even under optimal conditions, only 8.3\% are sequestered and stored for over 100 years, with the average being a mere 0.4\%.\\
Other issues include (1) the fact that OF is a geoengineering technique with global implications, and as such, can negatively impact ocean food webs and alter biochemical cycles \parencite{Chisholm2001Dis-CreditingFertilization, Zeebe2005FeasibilityLevels}, (2) the potential for OIF to cause "nutrient robbing", which means that macronutrients are drawn from other areas of the ocean, reducing the amount of biomass growth there, and leading to limited additionality \parencite{Zeebe2005FeasibilityLevels}, and (3) the emission of N\textsubscript{2}O (which is a potent greenhouse gas) caused by the usage of nitrate fertilizer. All of these issues have the potential to severely limit the net carbon removal effect of OF and cause unintended consequences for the ocean ecosystem.
Finally, measuring the effect of OF is difficult \parencite[87]{NationalAcademiesofSciences2022ASequestration}. While surface waters can be sampled for nutrients and phytoplankton growth and monitored using satellite imaging, measuring sequestration rates in the deep ocean is more challenging and requires highly advanced sensors and instruments \parencite{Chisholm2001Dis-CreditingFertilization, NationalAcademiesofSciences2022ASequestration}.
\subsection*{Cost}
Optimistic estimates for OIF claim costs as low as 2 US\$/t CO\textsubscript{2} removed \parencite{Fuentes-George2017ConsensusFertilization}, however, these estimates are at least partially influenced by the commercial and marketing interests of the companies involved. A more realistic estimate by \textcite{Harrison2013AOcean}, based on OIF experiements conducted in the past, includes the losses mentioned above and puts the cost of OIF at 18 US\$/t CO\textsubscript{2} in the best case, but admits that at the average long-term storage rate, they could be as high as 457 US\$/t CO\textsubscript{2}.
For OMF, a detailed economic analysis by \textcite{S.F.Jones2014TheNourishment} concluded that if only nitrogen fertilization is required, costs could be as low as 20 US\$/t CO\textsubscript{2} removed, including the cost for producing and transporting the nitrate. However, this study does not take into account the low rate of long-term sequestration.
All things considered, \textcite{Gattuso2021TheBeyond} expects the average cost per ton of CO\textsubscript{2} sequestered to be 230 US\$ across both approaches.
\section{Summary of Geochemical Strategies}
Geochemical CDR (gcCDR) strategies are based on enhancing natural geological carbon sinks on land and in the oceans on a global scale. Proponents of these strategies promise cheap, large-scale carbon removal. Their biggest advantage is the safety and durability of storage as either solid rock or sequestered in the ocean floor, which makes them less prone to leakage and requires less long-term care than the biological approaches discussed in Chapter 2. In combination, their sequestration potential could be between 5 and 10Gt CO2/y at proposed costs as low as 50US\$/t CO\textsubscript{2} for EW and 20US\$/t CO\textsubscript{2} for OF.\\
However, gcCDR strategies suffer from the global scale of deployment required to realize a significant impact, which presents logistical and environmental challenges. Enhanced weathering requires large-scale mining operations and large amounts of energy for processing and transportation, which is counter-intuitive to climate protection and would probably face opposition due to the impact on local ecosystems and communities. Also, as \textcite{Goll2021PotentialRock} conclude, it would only be viable if sufficient renewable energy was available to keep the supply chain carbon-neutral.
Ocean fertilization is highly dependent on uncontrollable factors such as ocean currents and ecosystem response and suffers from uncertain effectiveness for long-term sequestration and potential nutrient robbing. Therefore, more research into the real durability and additionality depending on the specific conditions of the ocean is needed.
From an economic perspective, the existing estimates for both approaches are highly uncertain and vary widely. They are mostly based on small-scale experiments and ecosystem models, making it difficult to accurately predict the cost and efficacy of large-scale implementation. The necessary mining and transportation also mean that the ongoing operating costs would be significantly higher compared to biological approaches. In a worst-case situation, if required investments and potential negative impacts are included, the cost per ton of CO\textsubscript{2} could exceed 400US\$.
Lastly, the global scale of these approaches, especially OF, requires international cooperation and a framework that governs who can deploy them in which locations and ensures proper accountability for the potential long-term consequences.
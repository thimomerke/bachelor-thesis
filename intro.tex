\mychapter{0}{Introduction}
Earth has been going through cycles of cold periods ("glacials") and warm periods ("interglacials") for at least the last one million years. However, since the beginning of the industrial revolution in the mid to late 18th century, an increasing amount of anthropogenic CO\textsubscript{2} emissions has impacted the natural carbon cycle. While slight more than half of the total emissions since 1850 have been absorbed by land and oceans, the remainder has accumulated in the atmosphere \parencite{Bergman2021TheJustice} and increased the greenhouse effect, causing the atmosphere to warm up and resulting in serious and at least partially irreversible damage to terrestrial and ocean ecosystems and both human and animal life \parencite[9]{IPCC2022SummaryPolicymakers}. Currently, anthropogenic emission amount to about 40 billion tons of CO\textsubscript{2} globally every year \parencite[4]{Friedlingstein2022Global2022}, caused by energy generation, but also agriculture, construction, and transportation. Despite the ongoing and increasing efforts and pledges to curb carbon emissions by transitioning to green, or net-zero, technologies, we are currently on track for a global average temperature increase of 2.8 degrees Celsius by the year 2100, according to estimates of the \textit{United Nations Environment Programme}. With the successful implementation of all current carbon emission reduction pledges, global warming could be limited to 2.4 degrees Celsius, but this is still well above the 1.5-degree goal set in the \textit{Paris Agreement} that needs to be achieved to limit the adverse impacts of climate change \parencite[35-36]{UNEP2022Emissions2022}. This anticipated overshoot, and the fact that some CO\textsubscript{2} emissions are hard to reduce to zero, necessitates implementing net-negative carbon removal (CDR) strategies. These include biological strategies based on land and ocean management, chemical processes such as advanced weathering, and mechanical and technological processes such as direct air capture (DAC) and bioenergy with carbon capture and storage (BECCS), all of which are intended to offset the remaining carbon emissions and compensate future emissions that exceed the remaining carbon budget.\\This thesis provides a holistic overview of the different approaches and compares them based on their theoretical and practical carbon-removal potential, limitations, and economic and environmental feasibility.